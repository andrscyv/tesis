\chapter*{Conclusiones}
\addcontentsline{toc}{chapter}{Conclusiones}

% Las conclusiones tampoco cuentan como capítulo

\noindent
Se logró implementar un sistema que satisface los requerimientos funcionales y
no funcionales dentro de las restricciones establecidas. El jugador PIMC es
capaz de elegir una ficha a tirar a partir del estado del juego proporcionado
como entrada al sistema. Con una elección de parámetros basada en búsqueda en
cuadrícula, vence en más del 70 \% de las veces a un equipo \textit{greedy},
incluso cuando forma parte de un equipo mixto. La misma elección de parámetros
asegura que el sistema toma aproximadamente un segundo en calcular su jugada.

Se ha demostrado que el método presentado es una solución factible para un
sistema de jugadores artificiales en una plataforma de dominó online. En la
estimación de capacidad se vió que una instancia de un servicio web basado en el
jugador PIMC sería capaz de atender desde cientos hasta unos cuantos miles de
usuarios al mes. Con el servidor de Digital Ocean que se utilizó para este
trabajo (con costo mensual de 100 pesos mexicanos aproximadamente), el costo por
usuario estaría por debajo de un peso al mes.

Otra contribución derivada de este trabajo se encuentra en el repositorio
mctspy, empleado para la implementación del jugador. Debido a que la biblioteca
original no ofrecía las mismas opciones de parametrización del algoritmo
utilizadas en este trabajo, se envió un \textit{pull request} para incluir la
parametrización por tiempo y número de simulaciones. Esta modificación fue
aceptada con éxito.

Existen distintas lineas de desarrollo que podrían basarse en este trabajo. En
primer lugar, se puede realizar la implementación del servicio web antes
aludido. Para esto sería necesario crear un aplicativo web en python e importar
el jugador PIMC como un paquete externo.

Otra linea de investigación es mejorar el rendimiento del jugador. La
incorporación de reglas expertas en el primer módulo del sistema, la
paralelización del algoritmo y el desarrollo de un mecanismo de inferencia para
estimar la distribución de las fichas desconocidas son algunas de las
oportunidades que se pueden explorar en futuros trabajos.