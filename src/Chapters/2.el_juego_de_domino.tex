\chapter{El juego de dominó}

\noindent

El dominó es un juego conocido alrededor de todo el mundo. Existen distintas
modalidades de juego que varían en el número de jugadores o en las reglas para
bajar una ficha. En esta sección se define el tipo de dominó del que trata este
trabajo. Posteriormente, se discuten algunas propiedades importantes del juego
que permiten hacer una estimación de su complejidad y compararlo con la
complejidad de otros juegos de mesa.


\section{Dinámica del juego}

Para jugar se utilizan veintiocho fichas, cada una marcada con dos números entre
el cero y el seis. Al iniciar la partida, se revuelven aleatoriamente las fichas
y se reparten siete a cada uno de los cuatro jugadores. Los jugadores se dividen
en dos equipos y se encuentran sentados en circulo, de tal forma que no hay dos
jugadores del mismo equipo adyacentes.

El jugador con la ficha de dos seises es el primero en tirar y el orden de las
tiradas sigue a la derecha. En cada turno, el jugador puede bajar una ficha si
tiene algún número en común con los extremos externos de las fichas que ya se
han jugado. El objetivo del juego es ser el equipo con el primer jugador que ha
bajado todas sus fichas o, en caso de que ningún jugador pueda bajar fichas, ser
el equipo cuyas fichas sumen el menor número de puntos.

\section{Factor promedio de ramificación}

Uno de los componentes de la complejidad del juego es el número promedio de
acciones que cada jugador puede tomar en su turno, también conocido como factor
promedio de ramificación. Un cálculo analítico de este factor es difícil, pues
depende de la repartición aleatoria al inicio del juego así como de la
estrategia de cada jugador. Para obtener una estimación del factor se realizó la
simulación de diez mil juegos con dos equipos de jugadores greedy y se obtuvo
los siguientes resultados.

\begin{figure}[ht]
    \begin{center}
        \includegraphics{factor_prom_ramificacion.png}
        \caption{Factor de ramificación promedio}
        En la gráfica se observa cómo varía el número de acciones posibles, en el
        eje vertical, conforme se avanza en los turnos del juego, en el eje
        horizontal. Las barras verticales de error representan el intervalo de
        confianza del 95\%.
        \label{FPR}
    \end{center}
\end{figure}

En la figura \ref{FPR} se muestra, en el eje vertical, el número promedio de
fichas que un jugador puede tirar en el turno correspondiente al eje horizontal.
El primer jugador siempre puede tirar cualquiera de sus fichas. Después de él,
los jugadores pueden tirar 3 o menos fichas en promedio. Este factor de
ramificación es pequeño si se compara al de otros juegos de mesa como el
ajedrez, el cual tiene un factor estimado entre 31 y 35 movimientos.

\section{Información imperfecta}

El segundo componente que influye en la complejidad del juego es la
incertidumbre asociada a las fichas que no se conocen. Para obtener una medida
de esta incertidumbre se calculó el número de configuraciones posibles de la
información escondida. Desde la perspectiva del primer jugador en tirar y
suponiendo que todos los jugadores bajan una ficha, se puede calcular el número
de formas distintas en que se pueden repartir las fichas que no se conocen para
cada vuelta del juego (cada vez que vuelve a ser turno del primer jugador ) y se
obtuvo la siguiente tabla.

\begin{table}[H]
    \centering
    \caption{Posibles configuraciones de las fichas que no se conocen}
    \label{PC}
    \begin{tabular}{|c|c|}
        \hline
        Vuelta & Posibles formas de repartir las fichas desconocidas \\
        \hline
        1      & 400 millones                                        \\
        2      & 17 millones                                         \\
        3      & 750 mil                                             \\
        4      & 34 mil                                              \\
        5      & 1600                                                \\
        6      & 90                                                  \\
        7      & 6                                                   \\
        \hline
    \end{tabular}

    % \begin{tabular}{c}
    % \footnotesize{Fuente: Bojilov y Phelps (2012).}
    % \end{tabular}

\end{table}

Tomando en cuanta ambos componentes de la complejidad, se concluye que la
incertidumbre asociada a la información imperfecta en las primeras vueltas del
juego muestra ser el mayor reto a superar.

% \subsection{Cambios cerebrales y genéticos}

% \noindent Brizendine (2010) escribe que algunos científicos piensan que ciertas áreas del cerebro son como centros de actividad que mandan señales eléctricas a otras áreas del cerebro ocasionando un determinado comportamiento.\footnote{ Por ejemplo, en el hombre la corteza del cíngulo anterior pesa opciones, detecta conflicto y motiva decisiones. La unión temporoparietal busca soluciones rápidas y ante situaciones estresantes toma en cuenta la perspectiva de otros individuos. La corteza cingulada anterior rostral se encarga de procesar los errores sociales, como la aprobación o desaprobación de otros.}

% \begin{quote}
%     \small{Mientras que la distinción entre los cerebros de niños y niñas empieza biológicamente, estudios recientes muestran que es \textit{solo} el comienzo. La estructura cerebral no está escrita sobre piedra en el nacimiento ni al final de la infancia, como antes se creía, sino que continúa cambiando a lo largo de la vida. Más que ser inmutable, nuestros cerebros son mucho más plásticos y cambiables de lo que los científicos creían hace una década. El cerebro humano es también la máquina de aprendizaje más talentosa que conocemos. Así que nuestra cultura y el cómo nos enseñaron a comportarnos desempeñan un papel importante en el diseño y reestructura de nuestros cerebros (Brizendine 2010, 5-6).}
% \end{quote}

% % Para citas muy largas es mejor el \begin{quote}

% \vspace{1em}
% \noindent \textbf{Hipótesis 3.} \hfill\begin{minipage}{\dimexpr\textwidth-3cm}
% \textit{La intensidad religiosa está relacionada negativamente con la innovación.}
% \end{minipage}
% \vspace{1em}

% Para plantear hipótesis


% Para diseñar tablas
