\chapter*{Introducción}
\addcontentsline{toc}{chapter}{Introducción}

% La introducción no cuenta como primer capítulo

\noindent 

En el primer capítulo se presentará el problema que se aborda en este trabajo. Se dará el 
contexto histórico de la relación entre los videojuegos y los jugadores artificiales para luego 
identificar el problema y definir tanto los objetivos como la metodología a seguir.

\section{Contexto}

La historia de los juegos por computadora inicia desde la década de 1950 en el ámbito 
académico y en los años setenta y ochenta gana popularidad para el público en general. Los 
videojuegos han tenido un gran impacto en la cultura popular, así como en grandes figuras 
de la computación que tuvieron su primer acercamiento a los ordenadores por medio de 
estos y del lenguaje BASIC

Asimismo, los juegos de mesa han tenido un papel importante en el desarrollo del área de 
inteligencia artificial siendo una área muy fructífera de investigación como en el caso del 
ajedrez y la famosa contienda entre Deep Blue y Garry Kasparov

\subsection{Identificación del problema}

 El desarrollo de videojuegos es un ámbito multidisciplinario en donde se utilizan técnicas 
de inteligencia artificial para complementar la experiencia de juego del usuario. Los 
jugadores artificiales (o bots) cumplen un papel importante como contrincantes o 
personajes secundarios dentro del juego. 

Con miras a desarrollar una versión online del juego de dominó con un modelo de 
monetización basado en anuncios, se tiene como uno de los objetivo maximizar el número 
de impresiones de los anuncios en el usuario. Así, es necesario proveer una experiencia 
atractiva que tenga como efecto que el usuario pase un largo tiempo activo en la página.

El lanzamiento a mercado de un juego multijugador online presenta distintos retos. Entre 
ellos, existe la necesidad de crear una base mínima de usuarios que permita tener un tiempo 
razonable de espera para poder encontrar una partida a la cual unirse. Una forma de 
solventar parcialmente este obstáculo, particularmente en las primeras fases del 
lanzamiento, es contar con jugadores artificiales que suplementen la falta de contrincantes 
humanos.

Así, es deseable contar con un jugador artificial que permita a los usuarios iniciar una 
partida aun en las circunstancias en que no cuenten con suficientes personas para completar 
los equipos. Al momento en que se realiza este escrito, no se ha encontrado una 
implementación de código abierto de un jugador artificial para el juego de dominó (con las 
reglas que se usan en latinoamérica) que cuente tanto con una licencia que permita su uso 
comercial así como una interfaz de programación diseñada para su integración a un juego 
de tiempo real con usuarios humanos.



\subsection{Objetivos}

Implementar un programa de computadora que sea capaz de jugar en una partida de dominó 
como parte de un equipo de dos participantes que compiten con dos contrincantes.

\subsection{Metodología}

Para la implementación del bot, se decidió utilizar la metodología de cascada debido a que 
el alcance y la funcionalidad del proyecto es relativamente pequeña.

\subsection{Organización del documento}
\begin{enumerate}
    \item Introducción
    \item Análisis de requisitos del software
    \item El juego de dominó
    \item Diseño del programa
    \item Implementación
    \item Validación
    \item Conclusiones
\end{enumerate}


% Se sugiere que el primer párrafo de cada sección no tenga sangría: \noindent
